% =================================================================================================
% 1. ОПРЕДЕЛЕНИЕ КЛАССА ДОКУМЕНТА
% =================================================================================================
\documentclass[14pt, a4paper]{extarticle}


% =================================================================================================
% 2. ПОДКЛЮЧЕНИЕ ПАКЕТОВ
% =================================================================================================

% -------------------------------------------------------------------------------------------------
% Шрифты и кодировка
% -------------------------------------------------------------------------------------------------
\usepackage{fontspec}       % Для работы с OpenType и TrueType шрифтами
\usepackage{courier}        % Шрифт Courier для листингов
\usepackage{upgreek}        % Греческие буквы в прямом начертании

% -------------------------------------------------------------------------------------------------
% Математика
% -------------------------------------------------------------------------------------------------
\usepackage{amsmath}        % Основные математические окружения
\usepackage{amsfonts}       % Математические шрифты
\usepackage{amssymb}        % Дополнительные математические символы
\usepackage{amsthm}         % Оформление теорем
\usepackage{mathtools}      % Расширение возможностей amsmath
\usepackage{icomma}         % "Умная" запятая в десятичных дробях

% -------------------------------------------------------------------------------------------------
% Графика, таблицы и плавающие объекты
% -------------------------------------------------------------------------------------------------
\usepackage{graphicx}       % Для вставки изображений
\usepackage{pdfpages}       % Для вставки страниц из PDF-файлов
\usepackage{longtable}      % Для создания таблиц, занимающих несколько страниц
\usepackage{float}          % Управление плавающими объектами (фигуры, таблицы)
\usepackage{caption}        % Настройка подписей к плавающим объектам
\usepackage{tcolorbox}      % Создание цветных рамок и блоков
\tcbuselibrary{listings, breakable, skins} % Использование библиотек для tcolorbox
\usepackage{listings}       % Для оформления листингов кода
\usepackage{xcolor}         % Управление цветами

% -------------------------------------------------------------------------------------------------
% Форматирование страниц и абзацев
% -------------------------------------------------------------------------------------------------
\usepackage[includefoot=true]{geometry} % Управление полями страницы
\usepackage{fancyhdr}       % Настройка колонтитулов
\usepackage[none]{hyphenat} % Отключение переносов слов
\usepackage{indentfirst}    % Красная строка для первого абзаца в секции
\usepackage{enumitem}       % Расширенная настройка списков
\usepackage{etoolbox}       % Инструменты для программирования макросов

% -------------------------------------------------------------------------------------------------
% Оглавление и ссылки
% -------------------------------------------------------------------------------------------------
\usepackage{hyperref}       % Создание гиперссылок в документе
\usepackage{tocloft}        % Настройка оглавления

% -------------------------------------------------------------------------------------------------
% Вспомогательные пакеты
% -------------------------------------------------------------------------------------------------
\usepackage{zref-totpages}  % Для получения общего количества страниц
\usepackage{chngcntr}       % Для изменения способа нумерации счетчиков


% =================================================================================================
% 3. НАСТРОЙКА ГЕОМЕТРИИ СТРАНИЦЫ И КОЛОНТИТУЛОВ
% =================================================================================================

% Установка полей страницы в соответствии с методичкой МИРЭА
\geometry{a4paper, left=3cm, right=1cm, top=2cm, bottom=2cm}

% Настройка колонтитулов
\pagestyle{fancy}
\fancyhf{}
\renewcommand{\headrulewidth}{0pt}
\fancyfoot[C]{\thepage}


% =================================================================================================
% 4. НАСТРОЙКА ШРИФТОВ
% =================================================================================================

\setmainfont[Ligatures=TeX]{Times New Roman}
\setmonofont{Courier New}


% =================================================================================================
% 5. ФОРМАТИРОВАНИЕ АБЗАЦЕВ И ИНТЕРВАЛОВ
% =================================================================================================

\sloppy                                  % Улучшает выравнивание по ширине, избегая выхода за поля
\linespread{1.4}                         % Полуторный межстрочный интервал
\setlength{\parindent}{1.25cm}           % Величина абзацного отступа
\setlength{\parskip}{0cm}                % Отсутствие дополнительного интервала между абзацами
\hbadness=10000                          % Убираем дебильные предупреждения о длине пробелов


% =================================================================================================
% 6. НАСТРОЙКА ЗАГОЛОВКОВ
% =================================================================================================

\usepackage{titlesec} % Пакет для форматирования заголовков

\titleformat{\section}
  {\fontsize{18}{18}\bfseries}
  {\thesection}
  {0.5em}
  {}
  []
\titlespacing{\section}{1.25cm}{*0}{12pt}

\titleformat{name=\section,numberless}
  {\centering\fontsize{18}{18}\bfseries}
  {}
  {0.5em}
  {}
  []
\titlespacing{name=\section,numberless}{1.25cm}{*0}{12pt}

\titleformat{\subsection}
  {\fontsize{16}{16}\bfseries}
  {\thesubsection}
  {0.5em}
  {}
  []
\titlespacing{\subsection}{1.25cm}{*0}{12pt}

\titleformat{\subsubsection}
  {\fontsize{14}{14}\bfseries}
  {\thesubsubsection}
  {0.5em}
  {}
  []
\titlespacing{\subsubsection}{1.25cm}{24pt}{12pt}

% Исправление интервала перед подразделами
\newcommand{\subsecfix}{\vspace{24pt}}
\preto{\subsection}{\subsecfix}
\preto{\subsubsection}{\subsecfix}


% =================================================================================================
% 7. НАСТРОЙКА ОГЛАВЛЕНИЯ
% =================================================================================================

\renewcommand{\contentsname}{\makebox[\linewidth]{СОДЕРЖАНИЕ}}
\renewcommand{\cfttoctitlefont}{\fontsize{18}{18}\centering\bfseries}

\renewcommand{\cftsecleader}{\cftdotfill{.}}
\renewcommand{\cftsubsecleader}{\cftdotfill{.}}
\renewcommand{\cftsubsubsecleader}{\cftdotfill{.}}

\renewcommand{\cftsecfont}{\fontsize{14}{14}}
\renewcommand{\cftsecpagefont}{\fontsize{14}{14}}

\cftsetindents{section}{0em}{0em}
\cftsetindents{subsection}{0em}{0em}
\cftsetindents{subsubsection}{0em}{0em}

\setlength{\cftbeforesecskip}{0pt}

\renewcommand{\cftsecnumwidth}{2em}
\renewcommand{\cftsubsecnumwidth}{2.5em}
\renewcommand{\cftsubsubsecnumwidth}{3em}


% =================================================================================================
% 8. ФОРМАТИРОВАНИЕ СПИСКОВ
% =================================================================================================

\renewcommand{\labelenumii}{\arabic{enumi}.\arabic{enumii}.}
\renewcommand{\labelenumiii}{\arabic{enumi}.\arabic{enumii}.\arabic{enumiii}.}
\renewcommand{\labelenumiv}{\arabic{enumi}.\arabic{enumii}.\arabic{enumiii}.\arabic{enumiv}.}

\setlength{\labelsep}{0.6cm}
\setlist[enumerate,itemize]{leftmargin=22.5mm}
\setlist{nolistsep}


% =================================================================================================
% 9. НАСТРОЙКА МАТЕМАТИЧЕСКИХ ОКРУЖЕНИЙ
% =================================================================================================

\numberwithin{equation}{section}
\addtolength{\jot}{1em}

\expandafter\def\expandafter\normalsize\expandafter{%
    \normalsize%
    \setlength\abovedisplayskip{14pt}%
    \setlength\belowdisplayskip{14pt}%
    \setlength\abovedisplayshortskip{14pt}%
    \setlength\belowdisplayshortskip{14pt}%
}


% =================================================================================================
% 10. НАСТРОЙКА ПОДПИСЕЙ К РИСУНКАМ И ТАБЛИЦАМ
% =================================================================================================

\renewcommand{\figurename}{Рисунок}
\renewcommand{\tablename}{Таблица}

% Настройка формата подписей
\DeclareCaptionFormat{image}{\fontsize{12}{12}\textbf{#1#2#3}}
\DeclareCaptionFormat{table}{\fontsize{12}{8}\textit{#1#2#3}}
\DeclareCaptionFormat{longtable}{\fontsize{12}{8}\textit{#1#2#3}}

% Применение настроек
\captionsetup[figure]{format=image, labelsep=endash, justification=centering, skip=0pt}
\captionsetup[table]{format=table, singlelinecheck=false, labelsep=endash, margin=0pt, labelfont=it, justification=raggedright, position=top}
\captionsetup[longtable]{format=longtable, singlelinecheck=false, labelsep=endash, skip=0pt, margin=0pt, labelfont=it, justification=raggedright, position=top}

\setlength{\belowcaptionskip}{-12pt} % Уменьшение отступа после подписи

% Нумерация рисунков и таблиц в пределах раздела (например, Рисунок 1.1)
\AtBeginDocument{
  \renewcommand{\thefigure}{\thesection.\arabic{figure}}
  \renewcommand{\thetable}{\thesection.\arabic{table}}
  \counterwithin{figure}{section}
  \counterwithin{table}{section}
}


% =================================================================================================
% 11. НАСТРОЙКА ЛИСТИНГОВ КОДА
% =================================================================================================

\renewcommand\lstlistingname{Листинг}
\renewcommand\lstlistlistingname{Листинги}

\lstdefinestyle{gostlisting}{
    basicstyle=\fontsize{10}{8}\ttfamily\selectfont, 
    breakatwhitespace=true,
    breaklines=true,
    captionpos=t,
    keepspaces=true,
    showspaces=false,
    showstringspaces=false,
    showtabs=false,
    stepnumber=10,
    tabsize=4,
    numberstyle=\tiny,
}

\DeclareCaptionFormat{listing}{\fontsize{12}{12}#1#2\textit{#3}}
\captionsetup[lstlisting]{format=listing, singlelinecheck=false, margin=0pt, labelsep=endash, labelfont=it, justification=raggedright, position=top}

% Нумерация листингов в пределах раздела
\AtBeginDocument{
  \renewcommand{\thelstlisting}{\thesection.\arabic{lstlisting}}
  \counterwithin{lstlisting}{section}
}

% 1. Создаем общий стиль для листингов по ГОСТ
% Стиль принимает 2 аргумента: #1 - Название, #2 - Метка
\tcbset{
  gostlistingstyle/.style n args={2}{
    breakable,
    listing only,
    skin=freelance,      % Гарантирует полные рамки для всех фрагментов
    % --- Общие настройки рамки и отступов ---
    sharp corners,
    colback=white,
    colframe=black,
    boxrule=0.5pt,
    before skip=28pt plus 6pt,
    boxsep = -2mm,
    % --- Отключаем стандартный заголовок и рисуем свой ---
    title={},
    % Подпись для первого фрагмента листинга
    extras unbroken and first={
      overlay={
        \node[
          anchor=south west,
          text=black,
          yshift=-3pt,
          xshift=-4pt,
        ]
        at (frame.north west)
        {\fontsize{12}{14.4}\itshape\selectfont Листинг \thelstlisting\ --- #1};
      }
    },
    % Подпись для последующих фрагментов
    extras middle and last={
      overlay={
        \node[
          anchor=south west,
          text=black,
          yshift=-3pt,
          xshift=-4pt,
        ]
        at (frame.north west)
        {\fontsize{12}{14.4}\itshape\selectfont Продолжение Листинга \ref{lst:#2}};
      }
    },
    % --- Нумерация и опции для пакета listings ---
    step and label={lstlisting}{lst:#2},
    listing options={style=gostlisting},
  }
}

% 2. Объявляем окружение, используя созданный стиль
% Аргументы {m m} передаются в стиль {gostlistingstyle={#1}{#2}}
\DeclareTCBListing{lstgost}{ m m }{
    gostlistingstyle={#1}{#2}
}

% 3. Объявляем команду для вставки из файла, используя тот же стиль
% Аргументы {m m m} -> первые два передаются в стиль, третий в listing file
\DeclareTCBInputListing{\inplstgost}{ m m m }{
    listing file={#3},
    gostlistingstyle={#1}{#2}
}

% =================================================================================================
% 12. ГИПЕРССЫЛКИ
% =================================================================================================

\hypersetup{
    colorlinks,
    citecolor=black,
    filecolor=black,
    linkcolor=black,
    urlcolor=blue
}


% =================================================================================================
% 13. ПОЛЬЗОВАТЕЛЬСКИЕ КОМАНДЫ
% =================================================================================================

\newcommand{\gosttableofcontents}{
    % \addtocontents{toc}{\protect\thispagestyle{empty}}
    % \fancyfoot{}
    \tableofcontents
    % \thispagestyle{empty}
    \clearpage
    % \fancyfoot[C]{\thepage}
}

\newsavebox{\imagebox}

\newcommand{\insertpicture}[4][\relax]{
  \begin{figure}[H]
    \centering
    \ifx#1\relax
      % --- РЕЖИМ АВТОМАТИЧЕСКОГО РАЗМЕРА ---
      % Пользователь не указал ширину, используем нашу логику.
      % Аргументы сдвигаются: #4=путь, #2=подпись, #3=метка
      
      % Помещаем изображение в "коробку", чтобы измерить его
      \sbox{\imagebox}{\includegraphics{#4}}
      \ifdim\wd\imagebox>\ht\imagebox
        \includegraphics[width=\textwidth, keepaspectratio]{#4}
      \else
        \ifdim\ht\imagebox>2\wd\imagebox
          \includegraphics[height=0.75\textheight, keepaspectratio]{#4}
        \else
          \includegraphics[height=0.4\textheight, keepaspectratio]{#4}
        \fi
      \fi
      \caption{#2}
      \label{fig:#3}
    \else
      \includegraphics[width=#1\textwidth, keepaspectratio]{#4}
      \caption{#2}
      \label{fig:#3}
    \fi
  \end{figure}
}

\newenvironment{gosttable}[3]{
  \vspace{18pt}
  \begin{longtable}{#3}
  \caption{#1\label{tab:#2}} % <--- Переместите \label сюда
  \vspace{-16pt}
  \\
  \endfirsthead
  \caption*{Продолжение таблицы \thetable}
  \vspace{-16pt}
  \endhead
}
{
  \end{longtable}
}